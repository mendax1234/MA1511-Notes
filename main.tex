\documentclass[math,code]{amznotes}
\setcounter{tocdepth}{2}  % Only show sections in the ToC
\usepackage[utf8]{inputenc}
\usepackage{amsmath}
\usepackage{amsfonts}
\usepackage{graphicx}
\usepackage{tikz}
\usepackage{etoolbox}
\usepackage{tabularx}
\usepackage{float} % Needed for [H] placement specifier
\usepackage{wrapfig} % Needed for wrapping figures

\graphicspath{ {./images/} }
\geometry{
    a4paper,
    headheight = 1.5cm
}

\patchcmd{\chapter}{\thispagestyle{plain}}{\thispagestyle{fancy}}{}{}

\theoremstyle{remark}
\newtheorem*{claim}{Claim}
\newtheorem*{remark}{Remark}
\newtheorem{case}{Case}

\begin{document}
\fancyhead[L]{
    Engineering Calculus
}
\fancyhead[R]{
    Lecture Notes
}
\tableofcontents

\chapter{Partial Derivatives}
\section{Functions of Several Variables}
\subsection{Functions of Two variables}
\begin{dfnbox}{Functions of Two Variables}{function-of-two-variables}
    A function $\symbfit{f}$ of two variables is a rule that assigns to each ordered pair of real numbers $(x,y)$ in a set $\symbfit{D}$ a {\color{red} \textbf{unique}} real number denoted by $f(x,y)$. The set $D$ is the {\color{red} \textbf{domain}} of $f$ and its {\color{red} \textbf{range}} is the set of values that $f$ takes on, that is
    \begin{displaymath}
        \left\{ \, f(x,y) \mid (x,y) \in \symbfit{D} \, \right\}
    \end{displaymath}
\end{dfnbox}
We often write $z=f(x,y)$ to make explicit the value taken on by $\symbfit{f}$ at the general point $(x,y)$. The variables $\symbfit{x}$ and $\symbfit{y}$ are {\color{red} \textbf{independent variables}} and $\symbfit{z}$ is the {\color{red} \textbf{dependent variable}}.
\begin{notebox}
    \begin{remark}
        Compare this with the notation $y=f(x)$ for functions of a {\color{red} \textbf{single variable}}.
    \end{remark}
\end{notebox}
\subsection{Visualization}
Let's take functions of two variables as an example.
\begin{exbox}{Arrow Diagram for Functions of Two Variables}{arrow-diagram}
    A function of two variables is just a function whose domain is a subset of $\R^2$ and whose range is a subset of $R$. Then, we can draw the arrow diagram like below:
    \begin{figure}[H]
    \centering
    \includegraphics[width=0.35\linewidth]{images/arrow-diagram.png}
    \caption{Arrow Diagram}
    \label{fig:arrow-diagram}
    \end{figure}
\end{exbox}
\subsection{Graph}
Knowing the figure \ref{fig:arrow-diagram}, we can reform the diagram into a three dimensional coordinate system. And here comes the definition of \textbf{graph}
\begin{dfnbox}{Graph of functions of two variables}{graph-of-function-of-two-variables}
    If $\symbfit{f}$ is a function of two variables within domain $\symbfit{D}$, then the {\color{red} \textbf{graph}} of $\symbfit{f}$ is the set of all points $(x,y,z)$ in $\R^3$ such that $\symbfit{z}=f(x,y)$ and $(x,y)$ is in $\symbfit{D}$.
\end{dfnbox}
\begin{notebox}
    \begin{remark}
        Usually we can think $\symbfit{z}$ and $\symbfit{f(x,y)}$ are the same.
    \end{remark}
\end{notebox}
Now, let's see an example about how the graph for a function of two variables looks like.
\begin{exbox}{Graph of functions of two variables}{graph}
    In this figure \ref{fig:graph}, the area $\symbfit{S}$ is the graph of a function of two variables.
    \begin{figure}[H]
        \centering
        \includegraphics[width=0.5\linewidth]{images/graph.png}
        \caption{Graph}
        \label{fig:graph}
    \end{figure}
\end{exbox}
\begin{notebox}
    \begin{remark}
        In this example, the graph of the function is a \textbf{surface}. Graph is a more general concept and we can think that \textit{surface} is just one kind of \textit{graph} of a function.
    \end{remark}
\end{notebox}
\subsection{Level Curves}
\begin{dfnbox}{Level Curve}{levelcurve}
    The {\color{red} \textbf{level curves}} of a function $\symbfit{f}$ of two variables are the curves with equations $f(x,y)=\symbfit{k}$, where $\symbfit{k}$ is a constant (in the range of $\symbfit{f}$).
\end{dfnbox}
\begin{notebox}
    \begin{remark}
        From the definition of \nameref{dfn:levelcurve}, we should notice that a {\color{red} \textbf{level curve}} is the set of points in the domain of $\symbfit{f}$ at which $\symbfit{f}$ takes on a given value $\symbfit{k}$. In other words, it is a curve in {\color{red} $\symbfit{xy}$-plane} that shows where the graph of $\symbfit{f}$ has height $\symbfit{k}$ (above or below the $\symbfit{xy}$-plane).
    \end{remark}
\end{notebox}
\begin{exbox}{Find Level Curves}{findlevelcurves}
    Sketch the level curves of the function $f(x,y)=6-3x-2y$ for the values $k=-6,0,6,12$ \newline
    {\color{blue} \textbf{Solution}}: We just need to solve $6-3x-2y=k$, where $k=-6,0,6,12$. And then we will get four equations of line. We just need to plot these four lines out.
    \begin{figure}[H]
        \centering
        \includegraphics[width=0.5\linewidth]{images/level-curve-example.png}
        \caption{Level Curve Examples}
        \label{fig:level-curve-examples}
    \end{figure}
    In this example, we have a function of two variables and its graph should be in $\R^3$, but since we need to find its level curves. To find level curves, we need to decrease the dimension by 1, so its level curves are in $\R^2$.
\end{exbox}
\subsection{Contour Curves}
In the functions of two variables, we've seen that its level curves must be in $\R^2$. But thinking it deeply, how does level curves come? Basically, it is the projection of \textbf{contour curves} onto the $\symbfit{xy}$-plane. So, what is the \textbf{contour plane}?
\begin{dfnbox}{Contour Curve}{contourcurve}
    For a function of two variables, denoted by $z=f(x,y)$. A horizontal plane $z=k$ may or may not intersect with the graph of $\symbfit{f}$ along a curve. If the curve exists, we say that the curve is a {\color{red} \textbf{contour curve of height $k$}}. If not, we say the contour curve doesn't exist at height $k$ for the function $\symbfit{f}$.
\end{dfnbox}
To visual it more directly, let's see the example below
\begin{exbox}{Contour Curves Example}{contour-curve-examples}
    In this figure \ref{fig:contour-curves-example}, the \textit{horizontal races} are the same as \textit{contour curves} we've discussed above.
    \begin{figure}[H]
        \centering
        \includegraphics[width=0.5\linewidth]{images/contour-curve-example.png}
        \caption{Contour Curves Example}
        \label{fig:contour-curves-example}
    \end{figure}
\end{exbox}
\begin{notebox}
    \begin{remark}
        Sometimes, contour curves are also called \textbf{contour maps}.
    \end{remark}
\end{notebox}
\subsection{Functions of Three of more variables}
It's very difficult to visualize a function $f$ of three variables by its graph, since that would lie in $\R^4$. However, we do gain some insight into $f$ by examining its {\color{red} \textbf{level surfaces}}, which are surfaces with equations $f(x,y,z)=k$, where $k$ is a constant.
\begin{exbox}{Find level surfaces}{find-level-surfaces}
    Find the level surfaces of the function
    \begin{displaymath}
        f(x,y,z)=x^2+y^2+z^2
    \end{displaymath}
    {\color{blue} \textbf{Solution}}: The level surfaces are
    \begin{displaymath}
        x^2+y^2+z^2=k, ~where~ k\geq 0
    \end{displaymath}
    These form a family of concentric spheres of the function.
\end{exbox}

\section{Limits and Continuity}
\subsection{Limits}
\begin{dfnbox}{Limits}{limits}
    Let $\symbfit{f}$ be a function of two variables whose domain $\symbfit{D}$ includes points arbitrarily close to $(a,b)$. Then we say that the {\color{red} \textbf{limits of $f(x,y)$} as $(x,y)$ approaches $(a,b)$} is $\symbfit{L}$ and we write
    \begin{displaymath}
        \lim\limits_{(x,y) \to (a,b)} f(x,y) = L
    \end{displaymath}
    if for every number $\epsilon > 0$ there is a corresponding number $\delta > 0$ such that if $(x,y) \in D$ and $0 < \sqrt{(x-a)^2+(y-b)^2} < \delta$, then $\mid f(x,y) - L \mid < \epsilon$
\end{dfnbox}
\begin{exbox}{Explanation of Limits Definition}{explanation-limits-definition}
    Basically, it says that the distance between $f(x,y)$ and $L$ can be made arbitrarily small by making the distance from $(x,y)$ to $(a,b)$ sufficiently small, but not $0$. Let's form its arrow diagram as below,
    \begin{figure}[H]
        \centering
        \includegraphics[width=0.5\linewidth]{images/limits-illustration.png}
        \caption{Limits Definition Explanation}
        \label{fig:limits-definition-explanation}
    \end{figure}
    From this figure \ref{fig:limits-definition-explanation}, we can see that if any small interval $(L - \epsilon, L + \epsilon)$ is given around $L$, then we can find a disk $D_\delta$ with center $(a,b)$ and radius $\delta > 0$ such that $\symbfit{f}$ maps all the points in $D_\delta$ \text{[except possibly $(a,b)$]} into the interval $(L - \epsilon, L + \epsilon)$
\end{exbox}
\subsubsection{Show that a limit doesn't exist}
Since in the definition of \nameref{dfn:limits}, it refers only to the \textbf{distance} between $(x,y)$ and $(a,b)$. It does not refer to the \textbf{direction} of approach. Therefore, if the limit exists, then $f(x,y)$ must approach the same limit \textbf{no matter how} $(x,y)$ approaches $(a,b)$. Thus, one way to show that $\lim_{(x,y) \to (a,b)} f(x,y)$ does not exist is to find {\color{red} \textbf{different paths}} of approach along which the function has different limits.
\begin{notebox}
    \begin{claim}
        If $f(x,y) \to L_1$ as $(x,y) \to (a,b)$ along a path $C_1$ and $f(x,y) \to L_2$ as $(x,y) \to (a,b)$ along a path $C_2$, where $L_1 \neq L_2$, then $\lim_{(x,y) \to (a,b)} f(x,y)$ does not exist.
    \end{claim}
\end{notebox}
\begin{exbox}{Limit does not exist}{limits-dne}
    Show that $\lim\limits_{(x,y) \to (0,0)} \frac{x^2-y^2}{x^2+y^2}$ does not exist \newline
    {\color{blue} \textbf{Solution}}:
    \begin{enumerate}
        \item Let's approach $(0,0)$ along the $x$-axis. On this path $y=0$, the function becomes $f(x,0)=\frac{x^2}{x^2}=1, \forall x \neq 0$ and thus
        \begin{displaymath}
            f(x,y) \to 1 \text{ as } (x,y) \to (0,0) \text{ along the }x \text{-axis.}
        \end{displaymath}
        \item We now approach along the $y$-axis by putting $x=0$. Then $f(0,y)=\frac{-y^2}{y^2}=-1, \forall y \neq 0$, so
        \begin{displaymath}
            f(x,y) \to -1 \text{ as } (x,y) \to (0,0) \text{ along the }y \text{-axis.}
        \end{displaymath}
        So, DNE!
    \end{enumerate}
\end{exbox}
\subsubsection{Properties of Limits}
Before we talk about the properties, let's give two definitions about the \textbf{polynomial function} and the \textbf{rational function}.
\begin{dfnbox}{Polynomial Function}{polynomial-function}
    A {\color{red} \textbf{polynomial function}} of two variables (or polynomial, for short) is a sum of terms of the form ($x^my^n$, where $\symbfit{c}$ is a constant and $\symbfit{m}$ and $\symbfit{n}$ are \textbf{non-negative integers}. For example,
    \begin{displaymath}
        p(x,y)=x^4+5x^3y^2+6xy^4-7y+6
    \end{displaymath}
\end{dfnbox}
\begin{dfnbox}{Rational Function}{rational-function}
    A {\color{red} \textbf{rational function}} is a ratio of two \nameref{dfn:polynomial-function}s. For example,
    \begin{displaymath}
        q(x,y)=\frac{2xy+1}{x^2+y^2}
    \end{displaymath}
\end{dfnbox}
Now, we state that
\begin{notebox}
    \begin{claim}
        To find the limit of any polynomial function, we can use \textbf{direct substitution}. For example,
        \begin{displaymath}
            \lim\limits_{(x,y) \to (a,b)} p(x,y) = p(a,,b)
        \end{displaymath}
    \end{claim}
\end{notebox}
and,
\begin{notebox}
    \begin{claim}
        To find the limit of any rational function, we have
        \begin{displaymath}
            \lim\limits_{(x,y) \to (a,b)} q(x,y) = \lim\limits_{(x,y) \to (a,b)} \frac{p(x,y)}{r(x,y)} = \frac{p(a,b)}{r(a,b)}=q(a,b)
        \end{displaymath}
        provided that $(a,b)$ is in the domain of $q$. If not, we need to use the definition about limits does not exist.
    \end{claim}
\end{notebox}
\subsection{Continuity}
\begin{dfnbox}{Continuity}{continuity}
    A function $\symbfit{f}$ of two variables is called {\color{red} \textbf{continuous}} at $(a,b)$ if
    \begin{displaymath}
        \lim\limits_{(x,y) \to (a,b)} f(x,y) = f(a,b)
    \end{displaymath}
    We say that $\symbfit{f}$ is {\color{red} \textbf{continuous}} on $\symbfit{D}$ if $\symbfit{f}$ is continuous at every point $(a,b)$ in $\symbfit{D}$.
\end{dfnbox}
\section{Partial Derivatives}
\subsection{First Order Partial Derivatives}
\subsubsection{Definition}
\begin{dfnbox}{First Order Partial Derivatives}{first-order-partial-derivative}
    If $\symbfit{f}$ is a function of two variables, its {\color{red} \textbf{partial derivative}} are the functions $f_x$ and $f_y$ defined by
    \begin{gather*}
        f_x(x,y)=\lim\limits_{h \to 0} \frac{f(x+h,y)-f(x,y)}{h} \\
        f_y(x,y)=\lim\limits_{h \to 0} \frac{f(x,y+h)-f(x,y)}{h}
    \end{gather*}
\end{dfnbox}
\subsubsection{Notation for Partial Derivatives}
If $z=f(x,y)$, we write
\begin{gather*}
    f_x(x,y)=f_x=\frac{\partial f}{\partial x}=\frac{\partial}{\partial x}f(x,y)=\frac{\partial z}{\partial x}=f_1=D_1f=D_xf \\
    f_y(x,y)=f_x=\frac{\partial f}{\partial y}=\frac{\partial}{\partial y}f(x,y)=\frac{\partial z}{\partial y}=f_2=D_2f=D_yf
\end{gather*}
\subsubsection{Rule for Finding Partial Derivatives of $z=f(x,y)$}
\begin{thmbox}{Rules for Finding First Order Partial Derivatives}{find-first-order-partial}
    \begin{enumerate}
        \item To find $f_x$, regard $\symbfit{y}$ as a constant and differentiate $f(x,y)$ with regard to $\symbfit{x}$.
        \item To find $f_y$, regard $\symbfit{x}$ as a constant and differentiate $f(x,y)$ with regard to $\symbfit{y}$.
    \end{enumerate}
\end{thmbox}
\subsubsection{Rules of Differentiation}
Since when doing partial differentiation, we are treating one variable as a constant (if we have two variables in total), so the rules of differentiation for single variable still applies. Let's recap them.
\begin{thmbox}{Rules of Differentiation}{rules-of-differentiation}
    \begin{gather*}
        (fg)' = f'g+g'f \text{ (product rule)} \\
        (\frac{f}{g})'=\frac{f'g-g'f}{g^2} \text{ (quotient rule)} \\
        (f \circ g)' = f'(g) \cdot g' \text{ (chain rule) } ~~\text{where}~~ (f \circ g)(x) = f(g(x)) 
    \end{gather*}
\end{thmbox}
\subsubsection{Geometric Meaning of First Order Partial Derivatives} \label{sec:geometric-meaing-of-first-order-partial-deri}
$f_x(a,b)$ and $f_y(a,b)$ can be interpreted geometrically as the {\color{red} \textbf{slopes}} of the tangent lines at $P(a,b,c)$ to the traces $C_1$ and $C_2$ of $\symbfit{S}$ in the plane $y=b$ and $x=a$
\begin{figure}[H]
    \centering
    \includegraphics[width=0.3\linewidth]{images/partial-derivative-geometric-meaning.png}
    \caption{Geometric Meaning of First order Partial Derivatives}
    \label{fig:geo-meaning-first-order-partial-derivative}
\end{figure}
\begin{exbox}{Implicit Partial Differentiation}{implicit-partial-diff}
    Find $\partial z / \partial x$ and $\partial z / \partial y$ if $\symbfit{z}$ is defined implicitly as a function of $\symbfit{x}$ and $\symbfit{y}$ by the equation
    \begin{displaymath}
        x^3+y^3+z^3+6xyz+4=0
    \end{displaymath}
    {\color{blue} \textbf{Solution}}:
    \begin{enumerate}
        \item We differentiate implicitly with regard to $\symbfit{x}$, being careful to treat $\symbfit{y}$ as a constant and $\symbfit{z}$ as a function (of $\symbfit{x}$)
        \begin{displaymath}
            3x^2+3z^2\frac{\partial z}{\partial x}+\underbrace{6yz+6xy\frac{\partial z}{\partial x}}_\text{Product Rule! Be careful!}=0
        \end{displaymath}
        Solving this equation, we get
        \begin{displaymath}
            \frac{\partial z}{\partial x}=-\frac{x^2+2yz}{z^2+2xy}
        \end{displaymath}
        \item Similarly, we implicitly differentiate with regard to $\symbfit{y}$, we will get
        \begin{displaymath}
            \frac{\partial z}{\partial y}=-\frac{y^2+2xz}{z^2+2xy}
        \end{displaymath}
    \end{enumerate}
\end{exbox}
\subsection{Higher Order Partial Derivatives}
If $\symbfit{f}$ is a function of two variables, then its partial derivatives $f_x$ and $f_y$ are also functions of two variables, so we can consider their partial derivatives $(f_x)_x, (f_x)_y, (f_y)_x, (f_y)_y$, which are called {\color{red} \textbf{second partial derivatives}} of $\symbfit{f}$. If $z=f(x,y)$, we use the following notation:
\begin{gather*}
    (f_x)_x=f_{xx}=f_{11}=\frac{\partial}{\partial x}(\frac{\partial f}{\partial x})=\frac{\partial ^2 f}{\partial x^2}=\frac{\partial ^2 z}{\partial x^2} \\
    (f_x)_y=f_{xy}=f_{12}=\frac{\partial}{\partial y}(\frac{\partial f}{\partial x})=\frac{\partial ^2 f}{\partial y \partial x}=\frac{\partial ^2 z}{\partial y \partial x} \\
    (f_y)_x=f_{yx}=f_{21}=\frac{\partial}{\partial x}(\frac{\partial f}{\partial y})=\frac{\partial ^2 f}{\partial x \partial y}=\frac{\partial ^2 z}{\partial x \partial y} \\
    (f_y)_y=f_{yy}=f_{22}=\frac{\partial}{\partial y}(\frac{\partial f}{\partial y})=\frac{\partial ^2 f}{\partial y^2}=\frac{\partial ^2 z}{\partial y^2}
\end{gather*}
A very useful Theorem,
\begin{thmbox}{Clairaut's Theorem}{clairaut-theorem}
    Suppose $\symbfit{f}$ is defined on a disk $\symbfit{D}$ that contains the point $(a,b)$. If the functions $f_{xy}$ and $f_{yx}$ are both continuous on $\symbfit{D}$, then:
    \begin{displaymath}
        f_{xy}(a,b)=f_{yx}(a,b)
    \end{displaymath}
    This theorem is sometimes called {\color{red} \textbf{The Mixed Derivative Theorem}}.
\end{thmbox}
\section{Tangent Planes and Linear Approximations}
\subsection{Tangent Planes}
\subsubsection{What is a tangent plane}
\begin{dfnbox}{Tangent Plane}{tangent-plane}
    Let $P(x_0,y_0,z_0)$ be a point on the surface $\symbfit{S}$ defined by the equation $z=f(x,y)$. We will see that the {\color{red} \textbf{tangent plane}} to $\symbfit{S}$ at $\symbfit{P}$ consists of {\color{red} \textbf{all}} tangent lines at $\symbfit{P}$ to curves that lie on $\symbfit{S}$ and pass through $\symbfit{P}$.
\end{dfnbox}
For example, let $\symbfit{C_1}$ and $\symbfit{C_2}$ be the curves obtained by intersecting the vertical planes $y=y_0$ and $x=x_0$ with the surface $\symbfit{S}$. Let $\symbfit{T_1}$ and $\symbfit{T_2}$ be the tangent lines to the curves $\symbfit{C_1}$ and $\symbfit{C_2}$ at point $\symbfit{P}$. Then the {\color{red} \textbf{tangent plane}} to the surface $\symbfit{S}$ at the point $\symbfit{P}$ is defined to be the plane that contains both tangent lines $\symbfit{T_1}$ and $\symbfit{T_2}$. (See figure \ref{fig:tangent-plane})
\begin{figure}[H]
    \centering
    \includegraphics[width=0.3\linewidth]{images/tangent-plane.png}
    \caption{Tangent Plane}
    \label{fig:tangent-plane}
\end{figure}
\subsubsection{Derive the equation of a tangent plane}
\begin{enumerate}
    \item We know that any plane passing through the point $P(x_0,y_0,z_0)$ has an equation of the form:
    \begin{displaymath}
        A(x-x_0)+B(y-y_0)+C(z-z_0)=0
    \end{displaymath}
    \item By dividing this equation by $\symbfit{C}$ and letting $a=-A/C, b=-B/C$, we can write it in the form:
    \begin{equation} \label{eq:plane-intermediate}
        z-z_0=a(x-x_0)+b(y-y_0)
    \end{equation}
    \item If equation \eqref{eq:plane-intermediate} represents the tangent plane at $\symbfit{P}$, then its intersection with the plane $y=y_0$ must be the tangent line $\symbfit{T_1}$. Setting $y=y_0$ in Equation \eqref{eq:plane-intermediate} gives
    \begin{displaymath}
        z-z_0=a(x-x_0) ~~~\text{where}~ y=y_0
    \end{displaymath}
    \item We recognize this as the equation (in point-slope form) of a line with slope $\symbfit{a}$. But from beforehand (see this section about \ref{sec:geometric-meaing-of-first-order-partial-deri} the geometric meaning of first order partial derivatives), we know that the slope of the tangent line $T_1$ is $f_x(a,b)$. Therefore,
    \begin{displaymath}
        a=f_x(a,b)
    \end{displaymath}
    \item Similarly, we can get
    \begin{displaymath}
        b=f_y(a,b)
    \end{displaymath}
\end{enumerate}
Now, let's see the {\color{red} \textbf{Equation of a Tangent Plane}}
\begin{thmbox}{Equation of a Tangent plane}{tangent-plane-equation}
    Suppose $\symbfit{f}$ has continuous partial derivatives. An equation of the tangent plane to the surface $z=f(x,y)$ at the point $P(x_0,y_0,z_0)$ is
    \begin{displaymath}
        z-z_0=f_x(x_0,y_0)(x-x_0)+f_y(x_0,y_0)(y-y_0)
    \end{displaymath}
\end{thmbox}
\begin{notebox}
    \begin{remark}
        Note that the similarity between the equation of a tangent plane and the equation of a tangent line:
        \begin{displaymath}
            y-y_0=f'(x_0)(x-x_0)
        \end{displaymath}
        This will help you memorize the \nameref{thm:tangent-plane-equation}
    \end{remark}
\end{notebox}
\subsubsection{Normal Vectors}
It can be shown that vector
\begin{displaymath}
    \left[\begin{array}{c}
        f_x(a,b) \\
        f_y(a,b) \\
        -1
    \end{array}\right]
\end{displaymath}
is normal to the tangent plane at $\symbfit{P}$. And this vector is called the {\color{red} \textbf{Normal Vector}} of the tangent plane at $\symbfit{P}$. Wish this knowledge, we can get the vector equations of our tangent plane and normal line
\begin{thmbox}{Tangent Plane (Vector Equation)}{tangent-plane-vector-equations}
    A vector equation of the tangent plane at $\symbfit{P}$ is
    \begin{displaymath}
        \symbfit{r} \cdot \left[\begin{array}{c}
             f_x(a,b) \\
             f_y(a,b) \\
             -1
        \end{array}\right]
        =
        \left[\begin{array}{c}
             a \\
             b \\
             f(a,b)
        \end{array}\right]
        \cdot
        \left[\begin{array}{c}
             f_x(a,b) \\
             f_y(a,b) \\
             -1
        \end{array}\right]
    \end{displaymath}
\end{thmbox}
\begin{thmbox}{Normal Lines}{normal lines}
    The equation of the normal line at $\symbfit{P}$ (which is the line passing through $\symbfit{P}$ and perpendicular to the tangent plane) is
    \begin{displaymath}
        \symbfit{r} =
        \left[\begin{array}{c}
             a \\
             b \\
             f(a,b)
        \end{array}\right]
        +
        \left[\begin{array}{c}
             f_x(a,b) \\
             f_y(a,b) \\
             -1
        \end{array}\right]
        t, ~~~t \in \symbfit{R}
    \end{displaymath}
\end{thmbox}
\subsection{Linear Approximations}
\begin{dfnbox}{Linearization}{Linearization}
    The linear function whose graph is this tangent plane, namely
    \begin{displaymath}
        L(x,y)=f(a,b)+f_x(a,b)(x-a)+f_y(a,b)(y-b)
    \end{displaymath}
    is called a {\color{red} \textbf{linearization}} of $\symbfit{f}$ at $(a,b)$ and the approximation,
    \begin{displaymath}
        f(x,y) \approx f(a,b)+f_x(a,b)(x-a)+f_y(a,b)(y-b)
    \end{displaymath}
    is called {\color{red} \textbf{linear approximation}} or the {\color{red} \textbf{tangent plane approximation}} of $\symbfit{f}$ at $(a,b)$.
\end{dfnbox}
\begin{notebox}
    \begin{remark}
        Note that the linear function $L(x,y)$ and the original function $\symbfit{f}$ are {\color{red} \textbf{not}} the same function!
    \end{remark}
\end{notebox}
\subsection{Differentiable function of two variables}
\begin{dfnbox}{Differentiable function of two variables}{differentiable-function-of-two-variables}
    If $z=f(x,y)$, then $\symbfit{f}$ is {\color{red} \textbf{differentiable}} at $(a,b)$ if $\Delta z$ can be expressed in the form
    \begin{displaymath}
        \Delta z = f_x(a,b)\Delta x+f_y(a,b)\Delta y + \epsilon_1\Delta_x + \epsilon_2\Delta_y
    \end{displaymath}
    where $\epsilon_1$ and $\epsilon_2$ are functions of $\Delta_x$ and $\Delta_y$ such that $\epsilon_1$ and $\epsilon_2 \to 0$ as $(\Delta_x,\Delta_y) \to 0$
\end{dfnbox}
Sometimes it is hard to use the Definition of \nameref{dfn:differentiable-function-of-two-variables} to directly check the {\color{red} \textbf{differentiability}} of a function, but below provides a convenient {\color{red} \textbf{sufficient}} condition for differentiability.
\begin{thmbox}{Differentiability Test}{differentiability-test}
    If the partial derivatives $f_x$ and $f_y$ exist near $(a,b)$ and are continuous at (a,b), then $f$ is differentiable at $(a,b)$
\end{thmbox}
\subsection{Differentials}
\begin{dfnbox}{Differential}{differential}
    For a differential function of two variables, $z=f(x,y)$, we define the {\color{red} \textbf{differentials}} $dx$ and $dy$ to be independent variables; that is, they can be given any values. Then the {\color{red} \textbf{differential}} $dz$, also called the {\color{red} \textbf{total differential}}, is defined by
    \begin{equation} \label{eq:two-varaibles-differentials}
        dz=f_x(x,y)dx+f_y(x,y)dy=\frac{\partial z}{\partial x}\cdot dx+\frac{\partial z}{\partial y}\cdot dy
    \end{equation}
\end{dfnbox}
\subsubsection{Geometric meaning of $dz$ and $\Delta z$}
Figure \ref{fig:geomertic-meaning-of-differential-two-variables} shows the geometric interpretation of the differential $dz$ and the increment $\Delta z$: $dz$  represents the change in height of the tangent plane, whereas $\Delta z$ represents the change in height of the surface $z=f(x,y)$ when $(x,y)$ changes from $(a,b)$ to $(a+\Delta x, b+\Delta y)$
\begin{figure}[H]
    \centering
    \includegraphics[width=0.5\linewidth]{images/geomertic-meaning-of-differential-two-variables.png}
    \caption{Geometric Meaning of Differentials (Two variables)}
    \label{fig:geomertic-meaning-of-differential-two-variables}
\end{figure}
\begin{exbox}{Use differentials to approximate}{differential-approximation}
    \begin{enumerate}
        \item If $z=f(x,y)=x^3+3xy-y^2$, find the differential $dz$
        \item If $x$ changes from $2$ to $2.05$ and $y$ changes from $3$ to $2.96$, compare the values of $\Delta z$ and $dz$
    \end{enumerate}
    {\color{blue} Solution}:
    \begin{enumerate}
        \item Definition of $dz$ gives that
        \begin{align*}
            dz&=\frac{\partial z}{\partial x}\cdot dx+\frac{\partial z}{\partial y}\cdot dy \\
            &=(2x+3y)\cdot dx+(3x-2y)\cdot dy  
        \end{align*}
        \item Putting $x=2, dx=\Delta x=0.05, y=3$ and $dy=\Delta y=-0.04$, we get
        \begin{align*}
            dz&=[2(2)+3(3)]\cdot 0.05+[3(2)-2(3)]\cdot (-0.04) \\
            &=0.65
        \end{align*}
        The increment of $z$ is:
        \begin{align*}
            \Delta z &= f(2.05,2.96)-f(2,3) \\
            &\approx0.6449
        \end{align*}
    \end{enumerate}
    Notice that $\Delta z \approx dz$ but $dz$ is easier to compute.
\end{exbox}
\subsubsection{How to memorize}
For a differential function of one variable, $y=f(x)$, we define the differential $dx$ to be an independent variable; that is $dx$ can be given the value of any real number. The differential of $y$ is then defined as:
\begin{equation} \label{eq:one-variable-differential}
    dy = f'(x)dx
\end{equation}
And its geometric interpretation can be shown in figure \ref{fig:geometric-meaning-of-differential-one-variable}
\begin{figure}[H]
    \centering
    \includegraphics[width=0.4\linewidth]{images/geometric-meaning-of-differential-one-variable.png}
    \caption{Geometric Meaning of Differential (One variable)}
    \label{fig:geometric-meaning-of-differential-one-variable}
\end{figure}
We can compare Equation \eqref{eq:one-variable-differential} with \eqref{eq:two-varaibles-differentials}, so that we can memorize the two variables one easily.
\end{document}